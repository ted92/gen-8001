\section{Introduction}
%todo: talk about main difficulties in open research data in our field
%todo: IDENTIFY these difficulties
Before the advent of scientific journals,
science was more oriented towards a closed approach, so that only a restricted number
of people could have access to it.
Many scientists, including Galileo, Newton, Kepler, and Hooke, made their discoveries into
something they could profit, and often papers were even encoded in anagrams.
Following this closed science logic though, it was difficult
to identify ownership and priority of scientific discoveries, leading even to
a debate on whom first discovered calculus between
Leibniz and Newton.

In modern times, scientific research is moving towards openness and free knowledge,
and ownership and priority issues are not the main concern anymore. However,
there are still research data, journals, and paper which are not accessible to all.
Data is not open due privacy and secrecy by for-profit organizations, and
papers are not open because they are published in pay-walled journals.
Merton\,\cite{merton1942science} arguments in favor of open science are
that, knowledge-creation is more efficient if scientists work together, and
it is morally binding on the professional scientist.

Besides knowledge and discoveries, science needs also to constantly proof itself
to be right, hence it needs to be reproducible. If scientists such as Newton, Kepler or Galileo,
could guarantee the reproducibility of their discoveries only by mathematical proofs,
in modern science, and especially after the advent of computers and information technology,
it is not possible anymore. The complexity and amount of data used
to formulate theorems and proofs has drastically changed the way of doing research,
and having an open knowledge approach is not good enough to perform good science.

Being our field of research computer science, and in particular
big data on blockchain technology and cryptocurrencies,
we want to enhance the main challenges of doing good scientific
research when research data and source code are not openly accessible.
Excessively closed research data can lead to reproducibility issues,
especially when machine learning and big data analysis are involved.
%
We will show in this manuscript the importance of open source code
and open research data when it comes to the scope of making good science.

%todo: insert this --> close models or data can slow down the scientific process, hence the progress

\section{Open Science}
Open science is the movement to make scientific research (i.e. data, software, publications)
accessible to all levels of an inquiring society, amateur or professional\,\cite{woelfle2011openscience}.
However, the term open science encapsulates many different concepts, such as: open educational resources,
open access, open peer review, open data, open source, and open methodology\,\cite{Knoth2015}.
%
Also, according to Fecher et al. (2014)\,\cite{Fecher2014}, there are different open science schools.
They can be for instance, oriented towards a pragmatic, public, or democratic approach, but
the purpose is commonly identifiable in opening up the process of knowledge creation,
making knowledge freely available for everyone, and enabling science to be accessible for citizens.
%
We focus in particular on the open access, and open data concepts, enhancing difficulties for
scientists, amateur and professionals to contribute in scientific development and innovation, if research data are not
openly accessible.

Our research topic involves blockchain technology and cryptocurrencies.
By definition of blockchain, research data are all
public and always available. If we then perform searches in a topic of our interest, using
public and open source web applications to explore research data, such as Dataverse\,\cite{crosas2011dataverse} or
Google Dataset Search\footnote{\url{https://toolbox.google.com/datasetsearch}}, we can
easily find the information we need about real-world implementation of blockchain
technology, such as Bitcoin\,\cite{nakamoto2009bitcoin}.
%
However, not all data in different fields of computer science
are easily accessible. If we perform research about privacy, or eHealth for instance,
we will not get the same amount of results, and most of the research data are not public.
%

%todo: continue here

%todo: check on google datasets search

\section{Privacy Preserving}
%todo: some data are sensitive and needs to be preserved and protected.

% todo: talk about privacy online and how open access can affect the resarch or reserachers in certain fileds
%todo: different open science schools\,\cite{Fecher2014}

%\newpage
% Bibliography
\bibliographystyle{acm}
\bibliography{bibliography}
